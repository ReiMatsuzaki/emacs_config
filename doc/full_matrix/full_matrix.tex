\documentclass[a4paper]{jsarticle}
\usepackage[dvipdfmx]{graphicx}
\usepackage{braket}
\def\vector#1{\mbox{\boldmath $#1$}}
\def\braces#1{\left( #1 \right)}
\def\bracem#1{\left\{ #1 \right\}}
\def\braceb#1{\left[ #1 \right]}
\def\lefts{\left(}
\def\rights{\right)}
\def\leftb{\left[}
\def\rightb{\right]}
\def\leftm{\left\{}
\def\rightm{\right\}}
\begin{document}
\title{}
\author{松崎 黎}
\maketitle



\section{概要}
Stewart JCP(1969)のSTO-NG展開の方法を複素STO-NG展開に拡張する。
 懸念事項は最適化の自由度が4N+1と非常に大きくなることである。

\section{実部および虚部の微分}
\subsection{一変数}
最初に、$f(z,z^*), z=x+iy$の場合に、$\partial f/\partial x$などを
計算する場合を考える。ただし、前提条件として、
\begin{eqnarray}
  f(z, z^*) \in \Re \\
\end{eqnarray}
つまり、
\begin{eqnarray}
  f(z, z^*) = f(z, z^*)^* = f(z^*, z)
\end{eqnarray}
を要請する。

一階微分を計算すると、
\begin{eqnarray}
  \frac{\partial f}{\partial x} &=&
  \frac{\partial z}{\partial x} \frac{\partial f}{\partial z} + \frac{\partial z^*}{\partial x} \frac{\partial f}{\partial z^*} 
  \\ &=&
  \frac{\partial f}{\partial z} + \frac{\partial f}{\partial z^*}
\end{eqnarray}

\begin{eqnarray}
  \frac{\partial f}{\partial y} &=&
  \frac{\partial z}{\partial y} \frac{\partial f}{\partial z} + \frac{\partial z^*}{\partial y} \frac{\partial f}{\partial z^*} 
  \\ &=&
  i \frac{\partial f}{\partial z} -i \frac{\partial f}{\partial z^*}
\end{eqnarray}
となる。つまり、単純に$z,z^*$の微分を計算すればいい。

次に、二階微分を計算する。
\begin{eqnarray}
  \frac{\partial^2 f}{\partial x^2} &=&
  \frac{\partial^2 f}{\partial z\partial z} +
  2\frac{\partial^2 f}{\partial z\partial z^*} +
  \frac{\partial^2 f}{\partial z^* \partial z^*} 
\end{eqnarray}
\begin{eqnarray}
  \frac{\partial^2 f}{\partial y^2} &=&
  -\frac{\partial^2 f}{\partial z\partial z} 
  +2\frac{\partial^2 f}{\partial z\partial z^*} 
  -\frac{\partial^2 f}{\partial z^* \partial z^*}   
\end{eqnarray}
\begin{eqnarray}
  \frac{\partial^2 f}{\partial x\partial y} &=&
  i\frac{\partial^2 f}{\partial z\partial z} 
  -i\frac{\partial^2 f}{\partial z^* \partial z^*}   
\end{eqnarray}
となる。

\subsection{多変数}

次に、多変数関数を扱う。
\begin{eqnarray}
  f(\bracem{z_i^*}, \bracem{z_i}) \in \Re \\
  z_i = x_i + i y_i
\end{eqnarray}
$a,b\in\bracem{x,y}$として、一階微分と二階微分を計算する。
\begin{eqnarray}
  \frac{\partial f}{\partial a_i} &=&
  \frac{\partial z_i}{\partial a_i} \frac{\partial f}{\partial z_i} + \frac{\partial z_i^*}{\partial a_i} \frac{\partial f}{\partial z_i^*} \\
  \frac{\partial^2 f}{\partial a_i \partial b_j} &=&
  \frac{\partial z_i}{\partial a_i}\frac{\partial z_j}{\partial b_j} \frac{\partial^2 f}{\partial z_i \partial z_j}  +
  \frac{\partial z_i}{\partial a_i}\frac{\partial z_j^*}{\partial b_j}\frac{\partial^2 f}{\partial z_i \partial z_j^*} +
  \frac{\partial z_i^*}{\partial a_i}\frac{\partial z_j}{\partial b_j}\frac{\partial^2 f}{\partial z_i^* \partial z_j} +
  \frac{\partial z_i^*}{\partial a_i}\frac{\partial z_j^*}{\partial b_j}\frac{\partial^2 f}{\partial z_i^* \partial z_j^*} 
\end{eqnarray}
ここで、$a_i$の微分は定数で$1,i,-i$のいずれかであることに注意。


\section{full-matrix least square}
\subsection{評価関数}

次の関数の極小点を探索する。
\begin{eqnarray}
  \varepsilon \braces{\bracem{c_i}, \bracem{z_i}, \lambda} = \|\chi-\varphi\|^2 + \lambda (1-\braces{\chi, \chi})(1-\braces{\chi^*, \chi^*})
\end{eqnarray}
ここで、c-productの意味で規格化された$\varphi$はfitting対象の関数で、
$\chi$は試行関数である。第二項は試行関数がc-productの意味で規格化される
ことを補修するために追加した項である。複素共役をかけているのは
評価関数を実に限定するためである。
試行関数の形として
\begin{eqnarray}
  \chi(\bracem{c_i}, \bracem{z_i})=\sum_i c_i u(z_i)
\end{eqnarray}
とおく。$\varepsilon$は先の条件を満たす実関数である。

\subsection{一階微分}
前の節の議論から、
$\mu \in \bracem{c_i, z_i}$および$\mu^* \in \bracem{c_i^*, z_i^*}$の微分が必要になる。
一階微分を計算する。
\begin{eqnarray}
  \begin{array}{ccl}
    \displaystyle \frac{\partial \varepsilon}{\partial \mu} &=&
    \displaystyle \Braket{\chi-\varphi, \frac{\partial \chi}{\partial \mu }} -2\lambda \braces{\chi, \frac{\partial \chi}{\partial \mu}}(1-\braces{\chi^*, \chi^*})   \\
    \displaystyle \braces{\frac{\partial \varepsilon}{\partial \mu}}^* &=& \displaystyle \frac{\partial \varepsilon}{\partial \mu^*} \\
    \displaystyle \frac{\partial \varepsilon}{\partial \lambda} &=& (1-\braces{\chi, \chi})(1-\braces{\chi^*, \chi^*})
  \end{array}
\end{eqnarray}

\subsection{二階微分}
二階微分を計算する。
\begin{eqnarray}
  \frac{\partial^2 \varepsilon}{\partial \mu \partial \nu} &=&
  \Braket{\chi-\varphi, \frac{\partial^2 \chi}{\partial \mu \partial \nu}}
  -2\lambda \bracem{
    \braces{\frac{\partial \chi}{\partial \nu}, \frac{\partial \chi}{\partial \mu}} +
    \braces{\chi, \frac{\partial^2 \chi}{\partial \mu \partial \nu}} } (1-\braces{\chi^*, \chi^*}) \\ 
  \frac{\partial^2 \varepsilon}{\partial \mu \partial \nu^*} &=&
  \Braket{\frac{\partial \chi}{\partial \nu}, \frac{\partial \chi}{\partial \mu }}
  +4\lambda \braces{\chi, \frac{\partial \chi}{\partial \mu}}
  \braces{\chi^*, \frac{\partial \chi^*}{\partial \nu^*}} \\
  \frac{\partial^2 \varepsilon}{\partial \mu^* \partial \nu^*} &=&
  \braces{\frac{\partial^2 \varepsilon}{\partial \mu \partial \nu}}^* \\
  \frac{\partial^2 \varepsilon}{\partial \mu \partial \lambda} &=&
  -2\braces{\chi, \frac{\partial \chi}{\partial \mu}}(1-\braces{\chi^*, \chi^*}) \\
  \frac{\partial^2 \varepsilon}{\partial \mu^* \partial \lambda} &=&
  \braces{\frac{\partial^2 \varepsilon}{\partial \mu \partial \lambda}}^* \\
  \frac{\partial^2 \varepsilon}{\partial \lambda^2} &=& 0
\end{eqnarray}

\subsection{微分基底}
行列要素の計算に必要な基底は以下で計算できる。
\begin{eqnarray}
  \begin{array}{ccc}
    \displaystyle
    \frac{\partial \chi}{\partial c_i} = u_i &
    \displaystyle
    \frac{\partial \chi}{\partial z_i} = c_iu_i' &  \\
    \displaystyle
    \frac{\partial^2 \chi}{\partial z_i \partial z_j} = \delta_{ij}c_iu_i'' &
    \displaystyle
    \frac{\partial^2 \chi}{\partial z_i \partial c_j} = \delta_{ij}u_i' &
    \displaystyle
    \frac{\partial^2 \chi}{\partial c_i \partial c_j} = 0
  \end{array}
\end{eqnarray}


\bibliographystyle{jplain}
\bibliography{library,books,paper}

\section{初期値}
最適化の初期値をどのように設定するのか。
少なくとも、軌道指数を決定した後ならば
展開系数は線形方程式から計算できるはず。
$\lambda =0$としてエラーの展開系数の微分だけを考える。
まずは、複素数としての微分は、
\begin{eqnarray}
  \frac{\partial \varepsilon}{\partial c_i} &=&
  \Braket{\chi - \phi, \frac{\partial \chi}{\partial c_i}} \\ &=&
  \Braket{\sum_j c_ju_j - \varphi, u_i} \\ &=&
  \sum_j c_j^* \Braket{u_j, u_i} - \Braket{\phi, u_i} 
\end{eqnarray}
このままでは見難いので、複素共役をとる。
\begin{eqnarray}
  \sum_j \Braket{u_i, u_j} c_j = \Braket{u_i, \phi}
\end{eqnarray}
つまり、線形方程式を解けばいい。

\end{document}